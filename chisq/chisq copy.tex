% !TeX spellcheck = en_US

\documentclass[11pt]{article} % default template
%% Packages
\usepackage{verbatim}
\usepackage[open,openlevel=1]{bookmark}
\usepackage{dirtytalk}
\usepackage{mathtools} %\DeclarePairedDelimiter
\usepackage{dsfont} %indicator
\usepackage{booktabs} % for fancy tables
\usepackage{makecell} % linebreak inside a cell
\usepackage{algorithm} % for algorithm box
\usepackage{algpseudocode} % for algorithm box
\usepackage{tikz}%for figures
\usepackage{subfigure} % subfigures
\usepackage{graphics}%for figures
\usepackage{amssymb}
\usepackage{latexsym}
\usepackage{epsfig}
\usepackage{amscd}
\usepackage{amsthm}
\usepackage{multirow} %table
\usepackage{adjustbox} % table size
\usepackage{float}
\usepackage{footmisc}
\usepackage{color}
\usepackage{enumerate}
\usepackage{bm}
\usepackage{setspace}
\usepackage{xparse}
\usepackage{tcolorbox}
\usepackage{lipsum}
\usepackage{blindtext}

\usepackage[titletoc,title]{appendix}
\usepackage[T1]{fontenc}

\usepackage[square,numbers]{natbib}
%\setcitestyle{numbers, square}

\usepackage{url}
\usepackage[font=small,labelfont=bf]{caption}
\captionsetup[table]{skip=0.5pt}
\usepackage{authblk} % authors
\usepackage{hyperref}[] % hyperlink
\hypersetup{
colorlinks = true,
linkcolor = blue, %Colour of internal links
filecolor = magenta,
urlcolor = blue, %Colour for external hyperlinks
citecolor = blue, %Colour of citations
pdfpagemode=UseOutlines
}
\usepackage{cleveref} % for hyperinks


\setlength{\textwidth}{\paperwidth}
\addtolength{\textwidth}{-6cm}
\setlength{\textheight}{\paperheight}
\addtolength{\textheight}{-4cm}
\addtolength{\textheight}{-1.1\headheight}
\addtolength{\textheight}{-\headsep}
\addtolength{\textheight}{-\footskip}
\setlength{\oddsidemargin}{0.5cm}
\setlength{\evensidemargin}{0.5cm}

%%%%%%%%%%%%%%%%%%%%%%%%%%%%%%%%%%%%%%%%%%%%%%%%%
%align page break
\allowdisplaybreaks[1]
\allowbreak

\newcommand\numberthis{\addtocounter{equation}{1}\tag{\theequation}}


% 1. basic math
% 2. norms
% 3. probability
% 4. minimax rates

% 1. basic math
\newcommand{\vecBold}[1]{\boldsymbol{#1}} %vector notation
\newcommand{\naturalNumber}{\mathbb{N}^{+}}
\newcommand{\real}{\mathbb{R}} %set of real numbers
\newcommand{\floor}[1]{\lfloor #1 \rfloor} %floor function
\newcommand{\bfloor}[1]{\Bigl\lfloor #1 \Bigr\rfloor} %big floor function
\newcommand{\indicator}[1]{\mathds{1} \left( #1 \right) }%indicator funcion
\newcommand{\sampleSize}{n}
%stat
\newcommand{\distparamMultinom}{\boldsymbol{p}}

% 2. norms and distances
\newcommand{\normProbVec}[3]{\|p_{#3}\|_{#1}^{#2}}
\newcommand{\distProbVec}[4]{\|p_{#3} - p_{#4}\|_{#1}^{#2}}
\newcommand{\Lone}{\mathbb{L}_1} %L1 norm
\newcommand{\Ltwo}{\mathbb{L}_2} %L2 norm
\newcommand{\Linfty}{\mathbb{L}_{\infty}} %inf norm
\newcommand{\vertiii}[1]{
	{\left\vert\kern-0.25ex\left\vert\kern-0.25ex\left\vert #1 
    \right\vert\kern-0.25ex\right\vert\kern-0.25ex\right\vert}
    }%triple norm notation
    
%lemma and theorem reference
\newcommand{\improvedTwoMomentMethod}{Lemma C.1 of \citet{Kim2022MinimaxTests}}
\newcommand{\normNegBesov}[2]{\|#1 \|^{#2}_{B^{-\gamma}_{2,2}}}
\newcommand{\normNegBesovTrunc}[2]{\widetilde{\|#1 \|^{#2}}_{B^{-\gamma}_{2,2}}}

% 3. probabilities
\newcommand{\mE}{\mathbb{E}} %expectation
\newcommand{\mP}{\mathbb{P}} %probability
\newcommand{\iid}{\stackrel{\text{i.i.d}}{\sim}} %iid


% 4. minimax
\newcommand{\minimaxTestingRateTwosample}{\rho^{\ast}_{n_1, n_2}}

\newcommand{\rvLap}{W}
\newcommand{\realizedLap}{w}

%parameters
\newcommand{\binNum}{\kappa}
\newcommand{\binNumOptimalTs}{\binNum_{(1)}}
\newcommand{\binNumOptimalIndep}{\binNum_{(2)}}
\newcommand{\discLapParam}{\zeta}
\newcommand{\smoothness}{s}
\newcommand{\ballRadius}{R}


%indexing
\newcommand{\sampleIndexY}{i}
\newcommand{\sampleIndexZ}{j}
\newcommand{\alphabetSize}{k}
\newcommand{\alphabetSizeY}{\alphabetSize_Y}
\newcommand{\alphabetSizeZ}{\alphabetSize_Z}
\newcommand{\dimDensity}{d}
\newcommand{\vectorIndex}{m}
\newcommand{\haarScaleSecondIndex}{v}
\newcommand{\adaptiveBinNumIndex}{t}
\newcommand{\mixtureIndex}{\boldsymbol{\eta}}
\newcommand{\adaptiveBinNumIndexSet}{[\gammamax]}

% prob, expectation, variance
\newcommand{\mEQPYPZ}{\mathbb{E}_{(Q_{P_Y}, Q_{P_Z})}}
\newcommand{\mEQPY}{\mathbb{E}_{Q_{P_Y}}}
\newcommand{\mEQPZ}{\mathbb{E}_{Q_{P_Z}}}


\newcommand{\mEQ}{\mathbb{E}_{P, Q}}
\newcommand{\mVQ}{\mathrm{Var}_{P,Q}}
\newcommand{\noiseVarIndep}[1]{\hat{\sigma}^{#1}_\alpha}


% Besov
\newcommand{\besovParamNorm}{p}
\newcommand{\besovParamMicroscope}{q}
\newcommand{\resLev}{j}
\newcommand{\primResLev}{J}
\newcommand{\primResLevOptimalTs}{J_{(1)}}
\newcommand{\wavFatherUnivIndex}{k}
\newcommand{\wavFatherIndex}{\boldsymbol{\wavFatherUnivIndex}}
\newcommand{\wavMotherUnivIndex}{\ell}
\newcommand{\wavMotherIndex}{\boldsymbol{\wavMotherUnivIndex}}
\newcommand{\wavMotherBooleanUnivIndex}{\epsilon}
\newcommand{\wavMotherBooleanIndex}{\boldsymbol{\wavMotherBooleanUnivIndex}}
\newcommand{\wavMotherBooleanIndexSet}{S_{\dimDensity}}
\newcommand{\wavCoef}{\theta}
\newcommand{\wavCoefFather}[1]{\wavCoef_{\primResLev, #1}}
\newcommand{\wavCoefFatherVec}[1]{\wavCoef_{#1}}
\newcommand{\wavCoefMother}[3]{ \wavCoef_{#1, #2}^{#3} }
\newcommand{\wavCoefMotherVec}[1]{\wavCoef_{#1}}
\newcommand{\besovSubscript}{\tilde{\mathcal{B}}_{\besovParamNorm,\besovParamMicroscope}^\smoothness}
%
\newcommand{\wavFatherFunc}{\phi}
\newcommand{\haarScaleIndices}[2]{\wavFatherFunc_{#1, #2}}
\newcommand{\multivInhomoWavFatherBasis}{\overline{\Phi}_{\primResLev}}
\newcommand{\wavGenericFatherCoef}{\wavCoef_{\wavFatherFunc}}
%
\newcommand{\wavMotherFunc}{\psi}
\newcommand{\wavMotherFuncPerturb}{\tilde{\psi}}
\newcommand{\haarMotherIndices}[3]{\wavMotherFunc_{#1, #2}^{#3}}
\newcommand{\multivInhomoWavMotherBasis}{\overline{\uppercase{\Psi}}_{\resLev}} %wavelet function basis subset at resolution level j
\newcommand{\wavGenericMotherCoef}{\wavCoef_{\wavMotherFunc}}

\newcommand{\besovCoeffVec}{{\theta}_{\resLev \cdot }}
%adaptive
\newcommand{\adaptiveSingleTest}{\Delta^{{\adaptiveBinNumIndex}}_ {\gamma/\gammamax}}
\newcommand{\adaptiveBinNumIndexProof}{\tau}


%upper bound
\newcommand{\hyperRec}[1]{\mathrm{H}_{#1}}
\newcommand{\binMNProbVecY}{\hat{p}_{\vecBold{Y}}}
\newcommand{\binMNProbVecZ}{\hat{p}_{\vecBold{Z}}}
\newcommand{\binMNProbVecYZ}{\hat{p}_{\vecBold{YZ}}}
%mathmatics

%dimensions


\newcommand{\genrr}{\mathcal{M}_{\texttt{GenRR}}}
%dimension-related
\newcommand{\vectorElements}[1]{
(#1_{1}, \ldots, #1_{\dimDensity})
}
\newcommand{\dimProd}[1]{\prod_{{#1}=1}^\dimDensity}
\newcommand{\kappaProd}{{\kappa}^\dimDensity}
\newcommand{\kappaProdHalf}{{\kappa}^{\dimDensity/2}}
\newcommand{\kappaProdInv}{{\kappa}^{-\dimDensity}}
\newcommand{\domainTs}{
[0,1]^{{\dimDensity}}
}
\newcommand{\domainIndep}{
[0,1]^{{\dimDensity_1+\dimDensity_2}}
}

%function spaces
\newcommand{\LinftySpace}{\mathbb{L}_{\infty}(\domainTs)}
\newcommand{\LtwoSpace}{\mathbb{L}_{2}(\domainTs)}


\newcommand{\besovBall}[2]{\tilde{\mathcal{B}}_{#1,#2}^\smoothness(\ballRadius)}

\newcommand{\holderTwosample}{\mathcal{C}^{\smoothness}} %new
\newcommand{\holderIndep}{\mathcal{H}^{\dimDensity_1 + \dimDensity_2}_\smoothness(\ballRadius)}
\newcommand{\holderBall}{\mathcal{C}^{\smoothness}(\ballRadius)} %new
\newcommand{\holderTwosampleArias}{\mathcal{H}^\dimDensity_\smoothness(\ballRadius)}

% density spaces
%Holder classes

\newcommand{\pBesovTs}{
	\mathcal{P}_{%subscript
		\text{Besov}
		}^{%superscript
		(\dimDensity, \smoothness, \besovParamMicroscope)
		}
}
\newcommand{\pBesovIndep}{
	\mathcal{P}_{%subscript
		\text{Besov}
		}^{%superscript
		(\dimDensity_1+\dimDensity_2, \smoothness, \besovParamMicroscope)
		}
}
\newcommand{\pHolderTwosample}{
	\mathcal{P}^{(\dimDensity, \smoothness)}_{\text{H\"{o}lder}}
	}
\newcommand{\pHolderIndep}{
	\mathcal{P}^{(\dimDensity_1 + \dimDensity_2, \smoothness)}_{\text{H\"{o}lder}}
	}




%%%%%%%%%%%%%%%%%%%%%%%%%%%%%%%%%%%%%%%%
%Besov



\newcommand{\bumpBasisBesovTwosample}[1]{
\psi^{\boldsymbol{\epsilon}^\ast}_{#1, \boldsymbol{\ell}}
}

\newcommand{\besovBasis}[3]{
\psi_{#1, \boldsymbol{#2}}^{\boldsymbol{#3}}
}

\newcommand{\besovCoeff}[3]{
\theta^{\boldsymbol{#3}}_{#1,\boldsymbol{#2}}
}
%besov indices
\newcommand{\besovEpsilonIndex}{
\{0,1\}^\dimDensity \backslash \{(0, \ldots, 0)\}
}

\newcommand{\besovLambda}[1]{
\{
0, 1, \ldots, (2^{#1} - 1)
\}^{\dimDensity}
}


%%%%%%%%%%%%%%%%%%%%%%%%%%%%%%%%%%%%%%%%
%Holder



\newcommand{\bumpBasisHolderTwosample}[1]{
\varphi_{(#1, \boldsymbol{\ell})}
}
%%%%%%%%%%%%%%%%%%%%%%%%%%%%%%%%%%%%%%%%


%U-statistic
\newcommand{\indepSigmaU}{
	\sum_{ (i_1, i_2, i_3, i_4) \in \mathbf{i}_4^n}
}

\newcommand{\indepSigmaUThree}{
	\sum_{ (i_1, i_2, i_3) \in \mathbf{i}_3^n}
}
%
\newcommand{\indepSigmaUTwo}{
	\sum_{ (i_1, i_2) \in \mathbf{i}_2^n}
}
%common
\newcommand{\norm}[1]{\left\lVert#1\right\rVert}





%%two sample testing

%domain



%kappa
\newcommand{\kappaTsHalf}[1]{
	\kappa^{#1d/2}
	}
\newcommand{\kappaTs}[1]{
	\kappa^{{#1}d}
	}



%privatized vectors

\newcommand{\yTildei}{\widetilde{Y}_{i}}
\newcommand{\zTildei}{\widetilde{Z}_{i}}
\newcommand{\yTildePrimei}{\widetilde{Y}'_{i}}
\newcommand{\zTildePrimei}{\widetilde{Z}'_{i}}


%two sample testing

% distribution space
\newcommand{\pTwosampleDisc}{\mathcal{P}^{(\alphabetSize)}}

% noise variance
\newcommand{\LapUParam}{\sigma_{\alpha}}
%% optimal kappa for two sample testing upper bound
\newcommand{\optimalKappaTwosample}{\kappa^\ast}

%privatized vectors
\newcommand{\elemRandomPrivTwoSampleY}{{\Phi}_{i,\vectorIndex}}
\newcommand{\elemRandomPrivTwoSampleZ}{{\Psi}_{j,\vectorIndex'}}
\newcommand{\elemRandomPrivTwoSampleYNumber}[2]{{\Phi}_{#1, #2}}
\newcommand{\elemRandomPrivTwoSampleZNumber}[2]{{\Psi}_{#1, #2}}
\newcommand{\elemFixedPrivTwoSampleY}{{\phi}_{i, \vectorIndex}}
\newcommand{\elemFixedPrivTwoSampleZ}{{\psi}_{j, \vectorIndex'}}


\newcommand{\vecRandomPrivTwoSampleY}{\boldsymbol{\Phi}_{i}}
\newcommand{\vecRandomPrivTwoSampleZ}{\boldsymbol{\Psi}_{j}}
\newcommand{\vecFixedPrivTwoSampleY}{\boldsymbol{\phi}_{i}}
\newcommand{\vecFixedPrivTwoSampleZ}{\boldsymbol{\psi}_{j}}
\newcommand{\vecFixedPrivTwoSampleYNumber}[1]{\boldsymbol{\phi}_{#1}}
\newcommand{\vecFixedPrivTwoSampleZNumber}[1]{\boldsymbol{\psi}_{#1}}
\newcommand{\vecRandomPrivTwoSampleYNumber}[1]{\boldsymbol{\Phi}_{#1}}
\newcommand{\vecRandomPrivTwoSampleZNumber}[1]{\boldsymbol{\Psi}_{#1}}



\newcommand{\vecRandomPrivAdaptiveY}{\tilde{\boldsymbol{\Phi}}_{i}}
\newcommand{\vecRandomPrivAdaptiveZ}{\tilde{\boldsymbol{\Psi}}_{j}}
\newcommand{\vecFixedPrivAdaptiveY}{\tilde{\boldsymbol{\phi}}_{i}}
\newcommand{\vecFixedPrivAdaptiveZ}{\tilde{\boldsymbol{\psi}}_{j}}

\newcommand{\elemRandomPrivAdaptiveY}{\tilde{\boldsymbol{\Phi}}_{i, \adaptiveBinNumIndex}}
\newcommand{\elemRandomPrivAdaptiveZ}{\tilde{\boldsymbol{\Psi}}_{j, \adaptiveBinNumIndex}}
\newcommand{\elemFixedPrivAdaptiveY}{ \tilde{\boldsymbol{\phi}}_{i, \adaptiveBinNumIndex, \vectorIndex}}
\newcommand{\elemFixedPrivAdaptiveZ}{\tilde{\boldsymbol{\psi}}_{j, \adaptiveBinNumIndex,\vectorIndex}}

\newcommand{\vecRandomPrivIndepY}{\boldsymbol{\Phi}_{i}}
\newcommand{\vecRandomPrivIndepZ}{\boldsymbol{\Psi}_{i}}


%U-statistic
\newcommand{\kernelTwoSample}{h_{ts}}
\newcommand{\kernelTwoSampleSym}{\bar{h}_{ts}}
\newcommand{\kernelIndep}{h_{in}}
\newcommand{\kernelIndepSym}{\bar{h}_{in}}







% two moments method
\newcommand{\momentTwosampleOneY}{M_{Y,1}(P)}
\newcommand{\momentTwosampleOneZ}{M_{Z,1}(P)}
\newcommand{\momentTwosampleYZ}{M_{Y Z,2}(P)}



\newcommand{\LDPviewSingleGeneric}[1]{\tilde{X}_#1}
\newcommand{\LDPviewsGeneric}{(\LDPviewSingleGeneric{1}, \ldots \LDPviewSingleGeneric{n})}
\newcommand{\LDPviewSingleGenericVec}[1]{\tilde{\vecBold{X}}_#1}
\newcommand{\LDPviewGenericVec}{(\LDPviewSingleGenericVec{1}, \ldots \LDPviewSingleGenericVec{n})}

\newcommand{\beforNoiseIndepY}{
	\widetilde{Y}_{i,k}
}

\newcommand{\beforNoiseIndepZ}{
	\widetilde{Z}_{i,j}
}


\newcommand{\priVecIndepRealization}{
\left(
	\phi_{i},
	\psi_{i}
\right)
}

\newcommand{\priVecIndepRV}{
\left(
	\Phi_{i, \kappa},
	\Psi_{i, \kappa}
\right)
}

\newcommand{\priVecIndepDataset}{
\left\{
	\priVecIndep
\right\}_{i=1}^n
}

%kappas
\newcommand{\kappaIndepHalf}[1]{
	\kappa^{\frac{{#1}(d_1+d_2)}{2}}
	}

\newcommand{\kappaIndep}[1]{
	\kappa^{{#1}(d_1+d_2)}
	}


%sigma
\newcommand{\sigmaIndep}[1]{
	\sigma_{\kappa, \alpha}^{#1}
}



%indicator functions
\newcommand{\indicatorIndepY}{
	\mathds{1}\bigl(\widetilde{Y}_i = k\bigr)
}

\newcommand{\indicatorIndepYPrime}{
	\mathds{1}\bigl(\widetilde{Y'}_i = k\bigr)
}

\newcommand{\indicatorIndepZ}{
	\mathds{1}\bigl(\widetilde{Z}_j = k\bigr)
}

\newcommand{\indicatorIndepZPrime}{
	\mathds{1}\bigl(\widetilde{Z'}_j = k\bigr)
}

\newcommand{\dimIndep}{
	\kappa^{d_1+d_2}
}

\newcommand{\scalingIndep}{
	\kappa^{\frac{d_1+d_2}{2}}
}


%lower bound proofs


% Adaptive tests
\newcommand{\gammamax}{
	\mathcal{N}
	}

\newcommand{\adaptiveTest}{
	\Delta_{\mathrm{adapt}}
}


% for lower bound
\newcommand{\mPPrivate}[1]{
	\mP_{{Q, {#1}}}^{(n)}
	}

\newcommand{\mEPrivate}[1]{
	\mE_{{Q, {#1}}}^{(n)}
	}
	
\newcommand{\mPNuRho}{
	\mP_{\nu_{\rho}}
	}
	


%%%%%%%%%%%%%%%%%%% SETS %%%%%%%%%%%%%%%%%%%
\newcommand{\sampleSets}[3]{\{{#1}_{#2}\}_{#2 \in [#3]}}
\newcommand{\laplaceSets}[4]{\{{#1}_{#2, #3}\}_{#3 \in [#4]}}

% set of distributions
\newcommand{\pNull}{\mathcal{P}_0}
\newcommand{\pAlterTwosample}{\mathcal{P}_1(\rhoTwosample)}
\newcommand{\pAlterIndep}{\mathcal{P}_1(\rho_{n})}

\newcommand{\rhoTwosample}{\rho_{n_1, n_2}}




% probability spaces
\newcommand{\sigmaAlgebraYPriv}{\mathcal{F}_{i}^\Phi}

% datasets
\newcommand{\datasetTwosampleY}{\mathcal{Y}_{n_1}}
\newcommand{\datasetTwosampleZ}{\mathcal{Z}_{n_2}}
\newcommand{\datasetIndep}{\mathcal{X}_{n}}
%algorithm names
\newcommand{\permuteTestTS}{\text{\textsc{PermuTestTS}}}
\newcommand{\permuteTestIndep}{\text{\textsc{PermuTestIndep}}}
\newcommand{\privatizer}{\text{\textsc{LapU}}}
\newcommand{\privatizerDisc}{\text{\textsc{DiscLapU}}}
\newcommand{\privatizerAdapt}{\text{\textsc{LapUAdapt}}}
\newcommand{\procedureTSDisc}{\text{\textsc{PrivUTS}}}
\newcommand{\procedureTSConti}{\text{\textsc{PrivUTS}}_\text{\textsc{H\"{o}lder}}}
\newcommand{\procedureTSAdapt}{\text{\textsc{PrivUTS}}_\text{\textsc{Adapt}}}
\newcommand{\procedureIndepMarginal}{ \text{\textsc{PrivUIndep}} }
\newcommand{\procedureIndepSplit}{ \text{\textsc{PrivUSplit}} }


%LapU


%binning function
\newcommand{\binner}[2]{ h^{(#1, #2)}_{\mathrm{bin}} }

% noise variances



\newcommand{\LapUDiscParam}{ p_{\alpha, c, \theta}}
\newcommand{\LapUIndepParam}{\sigma_{\alpha, c, \theta}}
\newcommand{\LapUIndepParamProof}{\sigma_{\alpha, 4, d^\ell}}
\newcommand{\sigmaAlphaKappa}[1]{
	\sigma_{\alpha, \kappa}^{#1}
}




% matrices
\newcommand{\yUmat}{[\gammamax]}
\newcommand{\zUmat}{\mathbf{L}}
\newcommand{\yUmatTilde}{\mathbf{\tilde{K}}}
\newcommand{\zUmatTilde}{\mathbf{\tilde{L}}}




%privatized vectors

\newcommand{\YPrivVec}[1]{\boldsymbol{\Phi}_{#1}}
\newcommand{\ZPrivVec}[1]{\boldsymbol{\Psi}_{#1}}


\newcommand{\YPriv}{\Phi_i}
\newcommand{\YPrivComp}[1]{\Phi_{i, #1}}
\newcommand{\YPrivOne}[1]{\Phi_{1, #1}}

\newcommand{\yPriv}{\phi_i}
\newcommand{\yPrivComp}[1]{\phi_{i, #1}}

\newcommand{\ZPriv}{\Psi_i}
\newcommand{\ZPrivComp}[1]{\Psi_{i, #1}}
\newcommand{\ZPrivOne}[1]{\Psi_{1, #1}}

\newcommand{\zPriv}{\psi_i}
\newcommand{\zPrivComp}[1]{\psi_{i, #1}}

\newcommand{\psionek}{\psi_{1k}}

%probability vector components
\newcommand{\pYk}{p_{Y}(k)}
\newcommand{\pZk}{p_{Z}(\vectorIndex')}
\newcommand{\pYZkk}{p_{YZ}(\vectorIndex, \vectorIndex')}
\newcommand{\pYpZkk}{p_{Y}(k) p_{Z}(\vectorIndex')}
\newcommand{\deltakk}{\Delta_{k,k'}}

%norms
\newcommand{\normpy}{\| p_{Y} \|_2}
\newcommand{\normpz}{\| p_{Z} \|_2}
\newcommand{\normpzsq}{\| p_{Z} \|_2^2}
\newcommand{\normdeltats}{\| p_{Y} - p_Z \|_2}
\newcommand{\normdeltatssq}{\| p_{Y} - p_Z \|_2^2}


%moments
\newcommand{\momentOneIndep}{M'_1(P)}

%independence testing


% definition of noise-added phi

\newcommand{\ik}{h_{in}}
\newcommand{\PsiIndepOne}{\Psi'_1}

%norms
\newcommand{\normpyz}{\|P_{\vecBold{YZ}}\|_2}
\newcommand{\normpyzsq}{\|P_{\vecBold{YZ}}\|_2^2}
\newcommand{\normpypz}{\|p_{Y}p_{Z}\|_2}
\newcommand{\normpypzsq}{\|p_{Y}p_{Z}\|_2^2}
\newcommand{\normdelta}{\|P_{\vecBold{YZ}} - p_Y p_Z\|_2}



% lower bound proofs
\newcommand{\intdone}{
\int_{[0,1]^{\dimDensity_1}}
	}
	
\newcommand{\intdtwo}{
\int_{[0,1]^{\dimDensity_2}}
	}

\newcommand{\intindepwhole}{\int_{[0,1]^{\dimDensity_1 + \dimDensity_2}}}	
\newcommand{\qDensityY}{
q^Y_i(\phi_i|y_i)
	}

\newcommand{\qDensityZ}{
		q^Z_i(\psi_i|z_i)
	}	
	
\newcommand{\qDensityYPrime}{
q^Y_i(\phi_i|y_i')
	}

\newcommand{\qDensityZPrime}{
		q^Z_i(\psi_i|z_i')
	}	

\newcommand{\qDensityYCdot}{
q^Y_i(\cdot|y_i)
	}

\newcommand{\qDensityZCdot}{
		q^Z_i(\cdot|z_i)
	}

\newcommand{\privateUnif}{
\tilde{f}_{0, i}
	}

\newcommand{\dMuY}{
d\mu^Y_i(\phi_i)
	}

\newcommand{\dMuZ}{
d\mu^Z_i(\psi_i)
	}






\newcommand{\rvY}{Y}
\newcommand{\rVecY}{\vecBold{\rvY}}
\newcommand{\rvZ}{Z}
\newcommand{\rVecZ}{\vecBold{\rvZ}}
\newcommand{\mT}{\mathcal{T}}
\newcommand{\mS}{\mathcal{S}}
\newcommand{\mV}{\mathcal{V}}
\newcommand{\tV}{\text{Var}}

\newcommand{\bX}{\textbf{X}}
\newcommand{\bY}{\textbf{Y}}
\newcommand{\bW}{\textbf{W}}
\newcommand{\bZ}{\textbf{Z}}
\newcommand{\tr}{\text{tr}}
\newcommand{\bcdot}{\boldsymbol{\cdot}}
\newcommand{\bell}{\boldsymbol{\ell}}



\newcommand{\convAS}{\overset{a.s}{\longrightarrow}}
\newcommand{\convP}{\overset{p}{\longrightarrow}}
\newcommand{\convD}{\overset{d}{\longrightarrow}}
\newcommand\fnurl[2]{%
\href{#2}{#1}\footnote{\url{#2}}%
}
\newcommand{\indep}{\rotatebox[origin=c]{90}{$\models$}} % independece
\DeclarePairedDelimiter\ceil{\lceil}{\rceil}
% \DeclarePairedDelimiter\floor{\lfloor}{\rfloor}


\setlength{\oddsidemargin}{0in} \setlength{\evensidemargin}{0in}
\setlength{\textwidth}{6.5in} \setlength{\topmargin}{-.2in}
\setlength{\textheight}{8.25in}
\newtheorem{definition}{Definition}[section]
\newtheorem{theorem}{Theorem}[section]
\newtheorem{lemma}{Lemma}[section]
\newtheorem{property}{Property}[section]
\newtheorem{proposition}{Proposition}[section]
\newtheorem{corollary}{Corollary}[section]
\newtheorem{remark}{Remark}[section]
\newtheorem{example}{Example}[section]

%comments
\newcommand{\ilmun}[1]{
	{ \color{blue} Ilmun: #1}
	}

\newcommand{\jongmin}[1]{
	{ \color{red} #1}
	}
	
\newcommand{\sw}[1]{
    { \color{green} S: #1}
    }
    
\newcommand{\sobolevTwo}[1]{\mathcal{W}_1^{#1}(\Omega)}
\newcommand{\advLossSobol}[1]{d_{#1}^\mathcal{W}}

\begin{document}

\begin{center}
\LARGE \bf
LDP two-sample  chi-squared test
\end{center}

\begin{comment}
\begin{keyword}
\kwd{local differential privacy}
\kwd{two-sample testing}
\kwd{independence testing}
\kwd{minimax separation rates}
\end{keyword}

%\end{frontmatter} #uncomment this to use the Bernoulli template
\end{comment}
%%%%%%%%% uncomment the codes above to use Bernoulli template (Part II)%%%%%%
%%%%%%%%%%%%%%%%%%%%%%%%%%%%%%%%%%%%%%%%%%%%%%%%%%%%%%%%%%%%%%%%%%%%%%%%%%%%%

\setcounter{tocdepth}{2}
\section{Setting}
\begin{itemize}
	\item $
\rVecY_i \stackrel{iid}{\sim}multi(\sampleSize_1, \distparamMultinom_{\rVecY}),
\rVecZ_i \stackrel{iid}{\sim}multi(\sampleSize_2, \distparamMultinom_{\rVecZ})$ with $\alphabetSize$ categories
	\item One-hot vector form i.e. random vectors with dependent Bernoulli random variable entries
	\item Allow for $n_1 \neq n_2$
	
\end{itemize}
\section{Privacy mechanism: Generalized Randomized Response}
\begin{definition}[Generalized Randomized Response~(Theorem 5.4. of \citet{Gaboardi2018LDPChisq})]
For a multinomial random vector 
$
\rVecY_i \stackrel{iid}{\sim}multi(\sampleSize_1, \distparamMultinom_{\rVecY})$, 
we define
\begin{align*}
\mP
\bigl(
\genrr(\rVecY_i) = \vecBold{y}'
|
\rVecY_i = \vecBold{y}
\bigr)
:=
\begin{cases}
\dfrac
	{\exp(\alpha)}
	{\exp(\alpha) + \alphabetSize - 1 }
	\text{ if }
	\vecBold{y}' = \vecBold{y}
	\\
\dfrac
	{1}
	{\exp(\alpha) + \alphabetSize - 1 }	
	\text{ if }
	\vecBold{y}' \neq \vecBold{y}.
\end{cases}
\end{align*}
Then  $\tilde{\rVecY}_i:=\genrr(\rVecY_i)$ is a multinomial random vector with probability vector
\begin{align*}
\tilde{\distparamMultinom}_{\rVecY}
:=
\distparamMultinom_{\rVecY}
	\frac
	{\exp(\alpha)}
	{\exp(\alpha) + \alphabetSize - 1 }
	+
(1-\distparamMultinom_{\rVecY})	
	\frac
	{1}
	{\exp(\alpha) + \alphabetSize - 1 }.
\end{align*}
\end{definition}
\noindent
Since $e^\alpha>1$ for $\alpha>0$, the probability of sending the original category is a little bit higher than sending the other category.
\citet{Gaboardi2018LDPChisq} constructs a private goodness-of-fit test based on a chi-square statistic evaluated on $\tilde{\rVecY}_i$'s.
They demonstrate that the limiting distribution is chi-square distribution both under the null and alternative.



\section{Two sample chi-square statistic}
We extend the goodness-of-fit test by \citet{Gaboardi2018LDPChisq} into two-sample testing by privatizing the raw samples
$\rVecZ_i \stackrel{iid}{\sim}multi(\sampleSize_2, \distparamMultinom_{\rVecZ})$
into $\tilde{\rVecZ}_j := \genrr(\rVecZ_j)$. 
Under the null, $\genrr(\rVecY_i)$ and $\genrr(\rVecZ_j)$ follow multinomial distributions with the same probability vector.
 Therefore, the usual two-sample chi-square test statistic
 	\begin{align*}
		T_{\chi} :=
		\sum_{\ell=1}^k
		\frac{
		\bigl(
		n_2
		\sum_{i=1}^{n_1}\tilde{\rVecY}_i(\ell)
		-
		n_1
		\sum_{j=1}^{n_1}\tilde{\rVecZ}_j(\ell)
		\bigr)^2
		}{
		n_1 n_2 (n_1 + n_2) 
		\sum_{j=1}^{n_1}
		\bigl(
		\tilde{\rVecY}_j(\ell)+\tilde{\rVecZ}_j(\ell)
		\bigr)
		}
	\end{align*}
  converges to a chi-square distribution with degree of freedom $\alphabetSize-1$ and yields a valid  test with size $\gamma$.
The test statistic is from Van der Vaart's book Asymptotic Statistics, pp. 253.

Minimax optimal Chi-square test~\citet{Arias-Castro2018RememberDimension} might not stay optimal when combined with $\genrr$.
The probability vectors has changed due to LDP, but they only affect the covariance matrix of the sample mean which does not affect the limiting distribution as long as it is rank $\alphabetSize-1$ projection matrix.

\section{Proof of Theorem~\ref{asymp:chisq} }

Tㅇe result is evident, since it is one of the exercise problems in chapter 17 of Van der Vaart's book, but...
\begin{itemize}
	\item Unequal sample size seems to cause trouble
	\item However, we may use Lyapuvov CLT which only assumes independence at the cost of some regularity condition
	\item Is there a multivariate version of Lyapuvov CLT?
	\item Van der Vaart's book does not introduce Lyapunov CLT, so there must be a solution that uses classical multivariate CLT. 
\end{itemize}
\begin{figure*}
	\includegraphics[width=0.8\textwidth]{van}
\end{figure*}
The proof of Theorem~\ref{asymp:chisq}  utilizes the following classical result.
\begin{theorem}[\citet{ferguson1996course}]
	If $\boldsymbol{X} \sim N(\boldsymbol{\mu}, \Sigma)$
	and
	$\Sigma$ is a projection matrix of rank $\nu$ and $\Sigma \boldsymbol{\mu} = \boldsymbol{\mu}$ then 
	$\boldsymbol{X}^\top \boldsymbol{X}  \sim \chi^2_\nu (\boldsymbol{\mu}^\top \boldsymbol{\mu})$.
\end{theorem}
We will use CLT, slutsky and continuous mapping theorem.
\subsection{Multivariate CLT}
Let $N:=n_1 + n_2$.
In order to apply CLT, 
we first express $n_2 \tilde{\rVecY}^{(\ell)} - n_1 \tilde{\rVecZ}^{(\ell)}$ as a sum of $N$ i.i.d.~$\alphabetSize$-dimensional random vectors as follows. For any $\ell \in [\alphabetSize]$,
\begin{itemize}
	\item $\tilde{\rVecY}_t^\ast(\ell) := \tilde{\rVecY}_t(\ell)$ and $\tilde{\rVecZ}_t^\ast(\ell) := 0$ for  $t=1,\ldots n_1$
	\item $\tilde{\rVecY}_t^\ast(\ell) := 0$ and $\tilde{\rVecZ}_t^\ast(\ell) := \tilde{\rVecZ}_{t-n_1}(\ell)$ for  $t=n_1+1,\ldots n_2$
\end{itemize}
Then we have $\tilde{\rVecY}^{(\ell)} = \sum_{t=1}^N$
		
we first calculate expectation and covariance matrix.
Note that $\tilde{\rVecY}^{(\ell)}$'s and $\tilde{\rVecZ}^{(\ell)}$'s are sum of Bernoulli random variables.
Under the null, first note that
$
	\mE[n_2 \tilde{\rVecY}^{(\ell)}] = 
	\mE[n_1 \tilde{\rVecZ}^{(\ell)}]
$
since
\begin{align*}
	\mE[n_2 \tilde{\rVecY}^{(\ell)}] &= n_1n_2\mE[\tilde{\rVecY}^{(\ell)}/n_1] = n_1 n_2
	\distparamMultinom_{\rVecY}(\ell) \text{ and}
	\\
	\mE[n_1 \tilde{\rVecZ}^{(\ell)}] &= n_1n_2\mE[\tilde{\rVecZ}^{(\ell)}/n_2] = n_1 n_2 \distparamMultinom_{\rVecZ}(\ell) = n_1 n_2 \distparamMultinom_{\rVecY}(\ell).
\end{align*}
Thus we can express the numerator of the test statistic as 
\begin{align*}
n_2 \tilde{\rVecY}^{(\ell)} - n_1 \tilde{\rVecZ}^{(\ell)}
&=
\bigl(
n_2 \tilde{\rVecY}^{(\ell)} - \mE[n_2 \tilde{\rVecY}^{(\ell)}]
\bigr)
-
\bigl(
n_1 \tilde{\rVecZ}^{(\ell)} - \mE[n_1 \tilde{\rVecZ}^{(\ell)}]
\bigr)
\\
&=
%express as independent sum
\sum_{i=1}^{n_1}
\bigl(
n_2 \tilde{\rVecY}_i(\ell)
-
\mE[n_2 \tilde{\rVecY}_i(\ell)]
\bigr)
-
\sum_{j=1}^{n_2}
\bigl(
n_1 \tilde{\rVecZ}_j(\ell)
-
\mE[n_1 \tilde{\rVecZ}_j(\ell)]
\bigr)
\\
&:=
%centered random variable as \check
\sum_{i=1}^{n_1}
\check{\rVecY}_i(\ell)
-
\sum_{j=1}^{n_2}
\check{\rVecZ}_j(\ell)
\end{align*}

\begin{align*}
	Var[
n_2
		\tilde{\rVecY}^{(\ell)}
		-
		n_1
		\tilde{\rVecZ}^{(\ell)}	
	]
&=
n_2^2
	Var[
		\tilde{\rVecY}^{(\ell)}]
+		
n_1^2
Var[
\tilde{\rVecZ}^{(\ell)}	
	]
	\\
	&=
n_2^2 n_1
\distparamMultinom_{\rVecY}(\ell)
\bigl(
1-\distparamMultinom_{\rVecY}(\ell)
\bigr)
+		
n_1^2 n_2
\distparamMultinom_{\rVecZ}(\ell)
\bigl(
1-\distparamMultinom_{\rVecZ}(\ell)
\bigr)
\\&=
n_1n_2(n_1+n_2)
\distparamMultinom_{\rVecY}(\ell)
\bigl(
1-\distparamMultinom_{\rVecY}(\ell)
\bigr).
\end{align*}
%
\begin{align*}
	&
	Cov[
n_2\tilde{\rVecY}^{(\ell)}
		-
		n_1
		\tilde{\rVecZ}^{(\ell)}
,
n_2\tilde{\rVecY}^{(\ell')}
		-
		n_1
		\tilde{\rVecZ}^{(\ell')}	
	]
\\&\stackrel{(1)}{=}
\mE
\bigl[
\bigl(
n_2\tilde{\rVecY}^{(\ell)}
		-
		n_1
		\tilde{\rVecZ}^{(\ell)}
\bigr)
\bigl(
n_2\tilde{\rVecY}^{(\ell')}
		-
		n_1
		\tilde{\rVecZ}^{(\ell')}
\bigr)
\bigr]
\\&=
n_2^2\mE[\tilde{\rVecY}^{(\ell)} \tilde{\rVecY}^{(\ell')}]
+
n_1^2\mE[\tilde{\rVecZ}^{(\ell)} \tilde{\rVecZ}^{(\ell')}]
%Y-Z terms exploit independence
-n_1^2 n_2^2
\distparamMultinom_{\rVecY}(\ell)
\distparamMultinom_{\rVecY}(\ell')
%
\\&\stackrel{(2)}{=}
n_2^2(n_1^2 - n_1)
\distparamMultinom_{\rVecY}(\ell)
\distparamMultinom_{\rVecY}(\ell')
+
n_1^2 (n_2^2 - n_2)
\distparamMultinom_{\rVecY}(\ell)
\distparamMultinom_{\rVecY}(\ell')
%Y-Z terms exploit independence
-n_1^2 n_2^2
\distparamMultinom_{\rVecY}(\ell)
\distparamMultinom_{\rVecY}(\ell')
\\&=
-n_1n_2(n_1+n_2)\distparamMultinom_{\rVecY}(\ell)
\distparamMultinom_{\rVecY}(\ell')
\end{align*}
For (1), we use the zero-mean property demonstrated in~\eqref{chisq_zero_mean}.
For (2), we use the following facts: a Bernoulli random variable in $\tilde{\rVecY}^{(\ell)}$ and another Bernoulli random variable in $\tilde{\rVecY}^{(\ell')}$ are independent between different sample, and dependent between the same sample. When between different samples, the expectation of the Bernoulli multiplication is  $\distparamMultinom_{\rVecY}(\ell)
\distparamMultinom_{\rVecY}(\ell')$.
 When between the same sample, the expectation of the Bernoulli multiplication is zero.
 
 Now we have to apply multivariate CLT to the following random vector:
 \begin{align*}
		\frac{
		n_2
		\sum_{i=1}^{n_1}
		\tilde{\rVecY}_i
		-
		n_1
		\sum_{j=1}^{n_2}
		\tilde{\rVecZ}^{(\ell)}
		}{
		\sqrt{n_1 n_2 (n_1 + n_2) }
		} 	
 \end{align*}
 
\subsection{aaa}
Under $H_0$. 
\begin{align*}
	(\sqrt{n/2})(H_1/n - H_2/n)
	\stackrel{d}{\to}
	N(0,\Sigma_p)
\end{align*}
Let $\Sigma_p= BDB^T$. We scale the statisic with
$\Sigma_p^{-1/2} \Pi$.
Then by the continuous mapping theorem,
\begin{align*}
	(\sqrt{n/2})
	\Sigma_p^{-1/2} \Pi
	(H_1/n - H_2/n)
	\stackrel{d}{\to}
	N(0, \Sigma_p^{-1/2} \Pi \Sigma_p  \Pi \Sigma_p^{-1/2})
\end{align*}
We can check that the variance of the limiting distribution, $\Sigma_p^{-1/2} \Pi \Sigma_p  \Pi \Sigma_p^{-1/2}$, is an identity matrix with a single zero on the diagonal.
\bibliographystyle{apalike}
\bibliography{reference}


\end{document}
